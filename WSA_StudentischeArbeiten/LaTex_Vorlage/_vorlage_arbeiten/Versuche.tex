\chapter{Versuche}

\section{Ist Zustand}

Im Moment werden die Pyrolysegase unter kontinuierlichem Fluss vom FTIR gemessen. Da die Pyrolyseprozesse jedoch sehr schnell von statten gehen und die Probengröße nicht größer als 15mg Brennstoff ist sind nur wenige Einzelmessungen möglich, bevor die Verbrennungsgase wieder aus der Messzelle gespült werden. So führt jede Messungen zu unterschiedlichen Ergebnissen, nicht nur in der Quantität der gefundenen Zusammensetzung, was aufgrund der anfänglichen zu und anschließenden Abnahme  der Verbrennungsgase in der Messzelle zu erwarten wäre, sondern auch in der Qualitativen Zusammensetzung des Gases. Da das Signal-Rausch-Verhältnis  zu schlecht ist, so dass einige Spezies, welche nur in geringen Mengen vorkommen oder nur schwache absorptive Eigenschafften haben mal vom Rauschen geschluckt werden, oder aber das Rauschen fälschlicher Weise als eine Spezies interpretiert wird. Da, das Signal-Rausch-Verhältnis jedoch mit dem Faktor Wurzel aus Anzahl der Messungen verbessert wird, soll das Verbrennungsgas in der Messzelle eingefangen werden und so eine höhere Anzahl an Messungen ermöglicht werden.


\section{Optimierung des Versuchsaufbaus}

Um das Verbrennungsgas in der Zelle einzufangen wurde am Ein- und Ablauf der Messzelle jeweils ein manuell zu betätigender Kugelhahn angebracht. Zusätzlich wurde ein ebenfalls, per manuellem Kugelhahn zu betätigender Bypass gelegt, durch den das weiterhin strömende Gas an der Messzelle vorbei abgeführt wird. Auf diese Weise wird ein zu starker Druckaufbau innerhalb des Reaktors, welcher zu einem Abbau des Wirbelschichtbetts führen könnte, vermieden. Ein zusätzlicher Vorteil dieses Aufbaus ist, dass der Reaktor während der Messungen bereits durchgespült wird. Mit Hilfe dieses Aufbaus wird Versucht, möglichst den Peak, der Pyrolysegase im FTIR einzufangen, um so mehrere kontinuierliche Messungen der Produkte zu ermöglichen. Auf diese Weise soll das SNR verbessert werden, so dass genauere, weniger durch Rauschen beeinflusste Messungen durchgeführt werden können. \\
\\